\documentclass{article}

\usepackage[utf8]{inputenc}
\usepackage[T1]{fontenc}
\usepackage[french]{babel}
\usepackage[fulloldstylenums]{kpfonts}
\usepackage[a5paper]{geometry}
\usepackage{sectsty}

\allsectionsfont{\mdseries}

\title{Méthodologie de la rédaction: l’utilisation de \LaTeX{} pour la rédaction de travaux, de mémoires et de thèses en philosophie}
\author{Christian Gagné}
\date{\today}

\newcommand\pkgname[1]{\emph{#1}}

\usepackage{hyperref}

\begin{document}

\maketitle

\tableofcontents

\section{Portée de cette présentation}

\section{Qu’est-ce qu’un logiciel libre?}

\section{Présentation de \TeX{} et \LaTeX{}: l’analogie des navigateurs Web et des suites bureautiques}

\subsection{\TeX{} est un moteur de disposition typographique}

\subsection{\LaTeX{} est un format de rédaction et de structure}

\section{Avantages de l’utilisation de \LaTeX{} pour la recherche et la rédaction en philosophie}

\subsection{Prise en charge des langues}

\subsection{Traitement de la bibliographie}

\subsection{Balisage sémantique et communication efficace d’une argumentation}

Concept de déictique (référence, contextualisation, voir Chrysippe): LaTeX permet une *deixis* plus aisée en proposant une syntaxe de *désignation* très compacte et concise. Parler des pratiques contemporaines de référence par hyperlien et accentuation dans les articles de blog ou de sites “Question and Answer”.

… Il y a la deixis et il y a la lexis, donc l’élément le plus typique et le plus répandu est l’accentuation (emphasis). Or, parfois, l’accentuation ou en général la *désignation de la lexis* coïncide effectivement avec la deixis et avec certains éléments d’analyse linguistique du discours.

Dès qu’une lexis est bien désignée et différenciée, c’est une victoire et un vecteur d’efficacité pour la communication d’une argumentation. Une telle désignation ne devrait pas être perçue comme purement esthétique (même si son aspect esthétique est lui-même hautement important): il s’agit d’illumination et de démonstration visible de la structure d’un discours argumentatif. Comme le dit Aristote, idéalement la lexis devrait peu importer, mais dans les faits on observe qu’elle est très importante.

Autre implication: une désignation adéquate facilite la *transformation* du texte. Or, tout texte composé à l’ordinateur est voué à la transformation. Il existe même un champ de l’informatique, soit la transformation de programmes, qui s’applique directement au texte argumentatif tel qu’encodé dans l’ordinateur.

Exemple: le paquet \pkgname{syllogism}.

\subsection{Publication aisée et normalisation}

En conclusion: matériellement, le texte est encodé selon les normes d’un format de rédaction à l’aide de mots-clés et, formellement, la désignation de la structure par un langage formel permet un traitement intelligent, reproductible et maîtrisable et une publication (communication) accessible, efficace, esthétique et *immédiate*.

\end{document}