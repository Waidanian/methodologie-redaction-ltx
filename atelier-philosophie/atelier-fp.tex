\documentclass{article}

\usepackage[utf8]{inputenc}
\usepackage[T1]{fontenc}
\usepackage[polutonikogreek,french]{babel}
\usepackage[fulloldstylenums]{kpfonts}
%\usepackage[a5paper]{geometry}
\usepackage{sectsty}
\usepackage{multicol}
\usepackage{fancyvrb}
\usepackage{tcolorbox}
\usepackage[%
    backend=biber,
    style=authortitle,]%
  {biblatex}

\addbibresource{atelier.bib}

\allsectionsfont{\mdseries}
\frenchbsetup{%
  StandardLayout=true,
}

\title{Méthodologie de la rédaction: l’utilisation de \LaTeX{} pour la rédaction de travaux, de mémoires et de thèses en philosophie}
\author{Christian Gagné}
\date{\today}

\newcommand\pkgname[1]{\emph{#1}}
\newcommand\entite[1]{{\textsc{#1}}}
\newcommand\opus[1]{{\textit{#1}}}
\newcommand\stephnum[1]{%
    \textsuperscript{#1}}

% Définition en exergue:
\newenvironment{defex}
{\begin{tcolorbox}[
    colback=yellow!5!white,
    colframe=red!25!black,
    coltext=red!25!black,
    arc=0mm,
    outer arc=0mm,
    toprule=0mm,
    bottomrule=0mm,
]}
{\end{tcolorbox}}

\usepackage{hyperref}

\begin{document}

\maketitle

\tableofcontents

\section{Portée de cette présentation}

Cette présentation s’adresse à toute personne ayant à rédiger des documents structurés ou à présenter une argumentation, c’est-à-dire \emph{beaucoup de gens}.

La finalité est de présenter \LaTeX{} en tant qu’outil de rédaction: montrer ses principes, son mode de fonctionnement, ses avantages et les résultats auxquels on parvient en l’utilisant.

Je mets l’accent sur la rédaction en sciences humaines et spécialement en philosophie, mais la méthode et les techniques présentées peuvent être bénéfiques pour tous les domaines (sciences de la nature, économie, création littéraire, pédagogie \emph{etc.}).

Comme le \emph{système} \LaTeX{} est constitué de logiciels libres, je présente d’abord ce que j’entend par logiciel libre et en quoi importe l’utilisation de logiciels libres et publics au lieu de logiciels privés.

Pour faire comprendre le fonctionnement de \LaTeX{} et le situer par rapport aux autres outils de rédaction, je procède par analogies avec des outils plus familiers: les navigateurs Web et les suites bureautiques.

Je présente les avantages de \LaTeX{} pour les trois étapes principales du travail documentaire: la recherche, la rédaction et la publication.

\section{Qu’est-ce qu’un logiciel libre?}

Les nombreux logiciels que nous utilisons aujourd’hui dans notre travail et nos loisirs appartiennent à deux grandes catégories méconnues: les logiciels libres ou de «source ouverte» et les logiciels privés.

\begin{description}

\item[Logiciel libre] Un logiciel libre est un logiciel développé en public et garantissant aux utilisateurs quatre libertés fondamentales:

\begin{enumerate}
\item La liberté d’utiliser le logiciel;
\item La liberté d’étudier le fonctionnement du logiciel et de le modifier, ce qui nécessite d’avoir accès aux \emph{sources} du logiciel;
\item La liberté de distribuer des copies du logiciel;
\item La liberté de distribuer des copies \emph{modifiées} du logiciel.
\end{enumerate}

Voici quelques exemples de logiciels libres très utilisés:

\begin{itemize}
\item Les navigateurs Web \entite{Mozilla Firefox} et \entite{Google Chrome};
\item Les suites bureautiques \entite{OpenOffice} et \entite{LibreOffice};
\item Le lecteur multimédia \entite{VLC};
\item Certaines composantes du système d’exploitation \entite{Mac~OS~X}.
\end{itemize}

\item[logiciel privé] Un logiciel privé est un logiciel \emph{développé en privé} (le public n’a pas accès aux \emph{sources} du logiciel), qu’on peut utiliser sous licence mais dont on ne peut pas étudier le fonctionnement (souvent, des clauses de la licence défendent cela explicitement), qu’on ne peut donc ni modifier ni distribuer soi-même.

Exemples de logiciels privés:

\begin{itemize}
\item \entite{Microsoft Windows}, \entite{Microsoft Office};
\item La plupart des composantes de \entite{Mac~OS~X};
\item \entite{EndNote};
\item \entite{Skype}, etc.
\end{itemize}

\end{description}

Il y a une distinction importante entre un logiciel gratuit et un logiciel libre: la plupart des logiciels libres sont gratuits, mais certains sont payants et il est tout à fait possible de les vendre avec l’intention de générer un profit. Par ailleurs, plusieurs logiciels sont gratuits mais privés (comme \entite{Skype}, dont le \emph{logiciel client} est gratuit et avec lequel on paie pour un service plutôt que pour le logiciel lui-même).

Étant donné ces distinctions, établissons le statut de \LaTeX{}:

\begin{defex}
\begin{itemize}
\item \LaTeX{} est un logiciel à la fois libre et gratuit, distribué en tant que composante d’un système plus vaste (le système \TeX{}) qui est lui-même libre et gratuit.

\item \TeX{} et \LaTeX{} sont développés et gérés surtout en milieu universitaire; leurs développements et améliorations correspondent directement aux besoins et aux découvertes de travailleurs universitaires.

\item Le système \TeX{} peut être installé grâce à ce qu’on appelle une \emph{distribution}. Les principales distributions sont gérées par des oganismes à but non lucratif et sont disponibles sur le Web.
\end{itemize}
\end{defex}

De nos jours, dans le monde universitaire (particulièrement en Europe, mais cela s’observe à travers le monde), l’opinion de plus en plus répandue est la suivante: les logiciels libres correspondent mieux aux besoins et aux visées des chercheurs, des étudiants et des travailleurs universitaires, autant sur le plan technique que sur le plan éthique. Pour ce qui est de l’aspect technique, j’espère vous convaincre de l’excellence de \LaTeX{}; pour ce qui est de l’aspect éthique, il faudrait en faire une démonstration argumentée en une autre occasion, mais vous pourrez constater par vous-mêmes qu’il est intuitif de choisir un système libre et gratuit pour la recherche et la rédaction en philosophie.

\section{Présentation de \TeX{} et \LaTeX{}: l’analogie des navigateurs Web et des suites bureautiques}

Je vais tenter de vous faire saisir la nature du système et la distinction qui existe entre

\begin{itemize}
\item \LaTeX{}, le format documentaire et
\item \TeX{}, le moteur qui utilise le format \LaTeX{} pour effectuer la composition typographique du document,
\end{itemize}

en procédant par analogies et exemples.

\subsection{\TeX{} est un moteur de disposition typographique}

Quand on navigue sur le Web et qu’on charge une nouvelle page, il se passe essentiellement les choses suivantes:

\begin{enumerate}
\item Grâce à une \emph{adresse} (un URL), on localise un \emph{document} qui se trouve sur un ordinateur (un serveur) quelque part dans le monde;
\item Ce document est rédigé (\emph{encodé}) dans un certain format: le format des pages Web s’appelle HTML. Notre navigateur Web vérifie s’il comprend le format du document;
\item Si le navigateur reconnaît qu’il s’agit bien d’un document HTML, alors le navigateur effectue la \emph{composition} et le \emph{rendu} du document selon les \emph{règles} du format HTML: on peut ensuite lire la page sur notre écran.
\end{enumerate}

Au regard de ces considérations sur le fonctionnement des navigateurs Web, on peut présenter \TeX{} ainsi:

\begin{defex}
Le moteur typographique \TeX{} est aux documents \LaTeX{} ce que le navigateur Web est aux documents HTML. C’est un moteur de composition et de rendu qui produit un document lisible et mis en forme selon certaines règles.
\end{defex}

Notez que les logiciels de traitement de texte qu’on connaît bien possèdent aussi leur propre moteur de composition. Cependant, dans le cas du traitement de texte, la distinction entre le format du document et sa composition est beaucoup moins visible, à moins d’ouvrir un document Word ou OpenOffice d’une façon spéciale et avec un autre genre de logiciel afin de voir ce qui se cache «en-dessous» de la représentation graphique du document. Cela m’amène à parler des caractéristiques spécifiques du format \LaTeX{}.

\subsection{\LaTeX{} est un format de rédaction et de structure}

Développons l’analogie avec les logiciels de traitement de texte.
Quand on écrit un texte dans Word ou OpenOffice, on choisit des styles dans un menu et, en les appliquant, on en voit le résultat immédiatement et dans la même fenêtre. Cependant, à l’interne, le document sur lequel on travaille contient ce qu’on appelle des \emph{balises}, comme dans une page Web. Ces balises sont cachées, mais elles sont bel et bien enregistrées dans le document: le moteur de composition de Word ou d’OpenOffice s’en sert pour effectuer la composition du document. Sans ces balises, le logiciel ne saurait pas quelle mise en forme appliquer à quelle partie du texte. Ces balises servent à indiquer non seulement l’apparence du document, mais aussi sa \emph{structure sémantique}, par exemple:

\begin{itemize}
\item Quelles parties du texte sont des titres?
\item Quelles parties sont des citations longues?
\item Quelle partie est une table des matières? Est-elle écrite à la main par l’auteur ou générée automatiquement par le logiciel à partir des titres balisés?
\end{itemize}

Au regard de ces considérations sur le fonctionnement des logiciels de traitement de texte, on peut présenter \LaTeX{} ainsi:

\begin{defex}
\LaTeX{} est au moteur \TeX{} ce que les \emph{styles} d’un logiciel de traitement de texte sont au \emph{moteur de rendu} d’un logiciel de traitement de texte.

L’utilisation du format \LaTeX{} procure au moteur \TeX{} les \emph{instructions} nécessaires à la mise en forme du document et à la désignation de la \emph{structure sémantique} du document.
\end{defex}

\section{Avantages de l’utilisation de \LaTeX{} pour la recherche et la rédaction en philosophie}

Établissons d’abord un aspect primordial du processus de rédaction avec \LaTeX{}:

\begin{defex}
En principe, l’éditeur de texte dans lequel on écrit est indépendant du moteur de composition typographique utilisé.

Je dis bien en principe, car plusieurs éditeurs offrent une certaine \emph{intégration} avec, d’une part, le moteur de composition et, d’autre part, le visualiseur du PDF produit par la composition.
\end{defex}

Cela signifie qu’on a du \emph{choix}. Pour apprendre à utiliser \LaTeX{}, l’éditeur le plus recommandé est \entite{TeXworks} sur Windows ou Linux et \entite{TeXShop} sur Mac~OS~X.

Je montre maintenant les \emph{commandes} qu’on doit écrire en premier quand on commence à rédiger un document \LaTeX{}:

\begin{defex}
\begin{Verbatim}[numbers=left,
                 numbersep=4pt]
\documentclass{article}
\usepackage{polyglossia}
\setmainlanguage{french}

\begin{document}

\end{document}
\end{Verbatim}
\end{defex}

Dès qu’on a écrit ça, on peut commencer à écrire le texte entre les commandes \Verb|\begin{document}| et \Verb|\end{document}| et composer le document avec le moteur \entite{XeTeX}, qui est un moteur \TeX{} prenant en charge un très grand nombre de langues anciennes et modernes. Notez que la ligne~1 de mon exemple indique que la classe \pkgname{article} est choisie: la classe article est aussi celle qui est utilisée pour la version \emph{handout} du présent document. On pourrait aussi indiquer e.g.~\pkgname{ulthese} pour utiliser la classe de l’Université Laval.

Nous allons présenter brièvement différents aspects avantageux de la rédaction dans ce contexte.

\subsection{Prise en charge des langues}

La ligne~2 de mon exemple indique qu’on utilise pour le traitement des langues le paquet~\pkgname{polyglossia}. La ligne suivante indique la langue principale du document: \Verb|\setmainlanguage{french}|. Cela rend le texte du document immédiatement conforme aux règles typographiques traditionnelles: \pkgname{polyglossia} applique même spécifiquement le \opus{Lexique des règles typographiques en usage à l’Imprimerie nationale}, alors on est entre bonnes mains. En lisant le présent document, vous pouvez constater que la disposition est excellente. En particulier, les césures sont correctement appliquées et la justification est de loin supérieure à celle des logiciels de traitement de texte (et même supérieure au résultat par défaut qu’on peut obtenir avec un logiciel de mise en page comme \entite{Adobe~InDesign}).

Tandis qu’on parle des logiciels de mise en page, je vous donne un \emph{scoop}:

\begin{defex}
Quand on utilise \LaTeX{}, on n’a nullement besoin d’un logiciel de mise en page: le document PDF produit par \LaTeX{} est immédiatement conforme aux normes professionnelles d’édition.
\end{defex}

Voyons l’exemple d’un texte en grec ancien et de sa traduction, mis côte à côte:

\begin{multicols}{2}
\setlength{\columnseprule}{.1pt}

\begin{otherlanguage}{polutonikogreek}
εὐλογία ἄρα καὶ εὐαρμοστία καὶ εὐσχημοσύνη καὶ εὐρυθμία \stephnum{400ε} εὐηθείᾳ ἀκολουθεῖ, οὐχ ἣν ἄνοιαν οὖσαν ὑποκοριζόμενοι καλοῦμεν ὡς εὐήθειαν, ἀλλὰ τὴν ὡς ἀληθῶς εὖ τε καὶ καλῶς τὸ ἦθος κατεσκευασμένην διάνοιαν.
\end{otherlanguage}

\columnbreak

Ainsi, l’excellence du discours et de l’harmonie, la grâce du geste et du rythme découlent de l’excellence du caractère \stephnum{400e}, non de ce que nous désignons ainsi par euphémisme et qui n’est qu’absence de réflexion, mais au contraire de cette réflexion authentique d’un caractère où s’allient le bien et le beau\autocite[\nopp III,~400d--e]{plat:rep:lrx}.
\end{multicols}

Les deux textes sont composés selon leurs règles typographiques propres. Il a simplement fallu ajouter au préambule du document la ligne suivante:

\begin{defex}
\begin{Verbatim}
\setotherlanguage[variant=poly]{greek}
\end{Verbatim}
\end{defex}

On peut ainsi écrire du grec polytonique en \emph{balisant} les sections en grec avec des commandes appropriées, soit \Verb|\begin{greek}| et \Verb|\end{greek}|.

\subsection{Traitement de la bibliographie}

\subsection{Balisage sémantique et communication efficace d’une argumentation}

Concept de déictique (référence, contextualisation, voir Chrysippe): LaTeX permet une *deixis* plus aisée en proposant une syntaxe de *désignation* très compacte et concise. Parler des pratiques contemporaines de référence par hyperlien et accentuation dans les articles de blog ou de sites “Question and Answer”.

… Il y a la deixis et il y a la lexis, donc l’élément le plus typique et le plus répandu est l’accentuation (emphasis). Or, parfois, l’accentuation ou en général la *désignation de la lexis* coïncide effectivement avec la deixis et avec certains éléments d’analyse linguistique du discours.

Dès qu’une lexis est bien désignée et différenciée, c’est une victoire et un vecteur d’efficacité pour la communication d’une argumentation. Une telle désignation ne devrait pas être perçue comme purement esthétique (même si son aspect esthétique est lui-même hautement important): il s’agit d’illumination et de démonstration visible de la structure d’un discours argumentatif. Comme le dit Aristote, idéalement la lexis devrait peu importer, mais dans les faits on observe qu’elle est très importante.

Autre implication: une désignation adéquate facilite la *transformation* du texte. Or, tout texte composé à l’ordinateur est voué à la transformation. Il existe même un champ de l’informatique, soit la transformation de programmes, qui s’applique directement au texte argumentatif tel qu’encodé dans l’ordinateur.

Exemple: le paquet \pkgname{syllogism}.

\subsection{Publication aisée et normalisation}

En conclusion: matériellement, le texte est encodé selon les normes d’un format de rédaction à l’aide de mots-clés et, formellement, la désignation de la structure par un langage formel permet un traitement intelligent, reproductible et maîtrisable et une publication (communication) accessible, efficace, esthétique et *immédiate*.

\printbibliography

\end{document}