\usepackage[utf8]{inputenc}
\usepackage[T1]{fontenc}
\usepackage[polutonikogreek,french]{babel}
\usepackage[fulloldstylenums]{kpfonts}
\usepackage{amsmath}
\usepackage{substitutefont}
\usepackage{syllogism}
\usepackage{sectsty}
\usepackage{multicol}
\usepackage{fancyvrb}
\usepackage{tcolorbox}
\usepackage[%
    backend=biber,
    style=authortitle,]%
  {biblatex}

\substitutefont{LGR}{jkposn}{artemisia}

\addbibresource{atelier.bib}

\allsectionsfont{\mdseries}
\frenchbsetup{%
  StandardLayout=true,
}

\newcommand\pkgname[1]{\emph{#1}}
\newcommand\entite[1]{{\textsc{#1}}}
\newcommand\opus[1]{{\textit{#1}}}
\newcommand\stephnum[1]{%
    \textsuperscript{#1}}

% Définition en exergue:
\newenvironment{defex}
{\begin{tcolorbox}[
    colback=yellow!5!white,
    colframe=red!25!black,
    coltext=red!25!black,
    arc=0mm,
    outer arc=0mm,
    toprule=0mm,
    bottomrule=0mm,
]}
{\end{tcolorbox}}
% Environnement pour les cadres Beamer:
\newenvironment{shown}
{\begin{frame}}
{\end{frame}}

\title{Méthodologie de la rédaction: l’utilisation de \LaTeX{} pour la rédaction de travaux, de mémoires et de thèses en philosophie}
\author{Christian Gagné}
\date{\today}

\usepackage{hyperref}

\begin{document}

\maketitle

\tableofcontents

\section{Portée de cette présentation}

Cette présentation s’adresse à toute personne ayant à rédiger des documents structurés ou à présenter une argumentation, c’est-à-dire \emph{un très grand nombre de gens}.

La finalité est de présenter \LaTeX{} en tant qu’outil de rédaction: montrer ses principes, son mode de fonctionnement, ses avantages et les résultats auxquels on parvient en l’utilisant.

Je mets l’accent sur la rédaction en sciences humaines et spécialement en philosophie, mais la méthode et les techniques présentées peuvent être bénéfiques pour tous les domaines (sciences de la nature, économie, création littéraire, pédagogie \emph{etc.}).

Comme le \emph{système} \LaTeX{} est constitué de logiciels libres, je présente d’abord ce que j’entend par logiciel libre et en quoi importe l’utilisation de logiciels libres et publics au lieu de logiciels privés.

Pour faire comprendre le fonctionnement de \LaTeX{} et le situer par rapport aux autres outils de rédaction, je procède par analogies avec des outils plus familiers: les navigateurs Web et les suites bureautiques.

Je présente les avantages de \LaTeX{} pour les trois étapes principales du travail documentaire: la recherche, la rédaction et la publication.

Le présent document existe sous deux formes: une présentation sur diapositives et un \emph{handout}. L’ensemble des documents est disponible sur le site Web \entite{GitHub} à l’adresse suivante: \\
\url{https://github.com/Waidanian/methodologie-redaction-ltx} \\
On trouvera là la toute dernière version et, éventuellement, d’autres ressources.

\section{Qu’est-ce qu’un logiciel libre?}

Les nombreux logiciels que nous utilisons aujourd’hui dans notre travail et nos loisirs appartiennent à deux grandes catégories méconnues: les logiciels libres ou de «source ouverte» et les logiciels privés.

\begin{description}

\item[Logiciel libre] Un logiciel libre est un logiciel développé en public et garantissant aux utilisateurs quatre libertés fondamentales:

\begin{enumerate}
\item La liberté d’utiliser le logiciel;
\item La liberté d’étudier le fonctionnement du logiciel et de le modifier, ce qui nécessite d’avoir accès aux \emph{sources} du logiciel;
\item La liberté de distribuer des copies du logiciel;
\item La liberté de distribuer des copies \emph{modifiées} du logiciel.
\end{enumerate}

Voici quelques exemples de logiciels libres très utilisés:

\begin{itemize}
\item Les navigateurs Web \entite{Mozilla Firefox} et \entite{Google Chrome};
\item Les suites bureautiques \entite{OpenOffice} et \entite{LibreOffice};
\item Le lecteur multimédia \entite{VLC};
\item Certaines composantes du système d’exploitation \entite{Mac~OS~X}.
\end{itemize}

\item[logiciel privé] Un logiciel privé est un logiciel \emph{développé en privé} (le public n’a pas accès aux \emph{sources} du logiciel), qu’on peut utiliser sous licence mais dont on ne peut pas étudier le fonctionnement (souvent, des clauses de la licence défendent cela explicitement), qu’on ne peut donc ni modifier ni distribuer soi-même.

Exemples de logiciels privés:

\begin{itemize}
\item \entite{Microsoft Windows}, \entite{Microsoft Office};
\item La plupart des composantes de \entite{Mac~OS~X};
\item \entite{EndNote};
\item \entite{Skype}, etc.
\end{itemize}

\end{description}

Il y a une distinction importante entre un logiciel gratuit et un logiciel libre: la plupart des logiciels libres sont gratuits, mais certains sont payants et il est tout à fait possible de les vendre avec l’intention de générer un profit. Par ailleurs, plusieurs logiciels sont gratuits mais privés (comme \entite{Skype}, dont le \emph{logiciel client} est gratuit et avec lequel on paie pour un service plutôt que pour le logiciel lui-même).

Étant donné ces distinctions, établissons le statut de \LaTeX{}:

\begin{defex}
\begin{itemize}
\item \LaTeX{} est un logiciel à la fois libre et gratuit, distribué en tant que composante d’un système plus vaste (le système \TeX{}) qui est lui-même libre et gratuit.

\item \TeX{} et \LaTeX{} sont développés et gérés surtout en milieu universitaire; leurs développements et améliorations correspondent directement aux besoins et aux découvertes de travailleurs universitaires.

\item Le système \TeX{} peut être installé grâce à ce qu’on appelle une \emph{distribution}. Les principales distributions sont gérées par des oganismes à but non lucratif et sont disponibles sur le Web.
\end{itemize}
\end{defex}

De nos jours, dans le monde universitaire (particulièrement en Europe, mais cela s’observe à travers le monde), l’opinion de plus en plus répandue est la suivante: les logiciels libres correspondent mieux aux besoins et aux visées des chercheurs, des étudiants et des travailleurs universitaires, autant sur le plan technique que sur le plan éthique. Pour ce qui est de l’aspect technique, j’espère vous convaincre de l’excellence de \LaTeX{}; pour ce qui est de l’aspect éthique, il faudrait en faire une démonstration argumentée en une autre occasion, mais vous pourrez constater par vous-mêmes qu’il est intuitif de choisir un système libre et gratuit pour la recherche et la rédaction en philosophie.

\section{Présentation de \TeX{} et \LaTeX{}: l’analogie des navigateurs Web et des suites bureautiques}

Je vais tenter de vous faire saisir la nature du système et la distinction qui existe entre

\begin{itemize}
\item \LaTeX{}, le format documentaire et
\item \TeX{}, le moteur qui utilise le format \LaTeX{} pour effectuer la composition typographique du document,
\end{itemize}

en procédant par analogies et exemples.

\subsection{\TeX{} est un moteur de disposition typographique}

Quand on navigue sur le Web et qu’on charge une nouvelle page, il se passe essentiellement les choses suivantes:

\begin{enumerate}
\item Grâce à une \emph{adresse} (un URL), on localise un \emph{document} qui se trouve sur un ordinateur (un serveur) quelque part dans le monde;
\item Ce document est rédigé (\emph{encodé}) dans un certain format: le format des pages Web s’appelle HTML. Notre navigateur Web vérifie s’il comprend le format du document;
\item Si le navigateur reconnaît qu’il s’agit bien d’un document HTML, alors le navigateur effectue la \emph{composition} et le \emph{rendu} du document selon les \emph{règles} du format HTML: on peut ensuite lire la page sur notre écran.
\end{enumerate}

Au regard de ces considérations sur le fonctionnement des navigateurs Web, on peut présenter \TeX{} ainsi:

\begin{defex}
Le moteur typographique \TeX{} est aux documents \LaTeX{} ce que le navigateur Web est aux documents HTML. C’est un moteur de composition et de rendu qui produit un document lisible et mis en forme selon certaines règles.
\end{defex}

Notez que les logiciels de traitement de texte qu’on connaît bien possèdent aussi leur propre moteur de composition. Cependant, dans le cas du traitement de texte, la distinction entre le format du document et sa composition est beaucoup moins visible, à moins d’ouvrir un document Word ou OpenOffice d’une façon spéciale et avec un autre genre de logiciel afin de voir ce qui se cache «en-dessous» de la représentation graphique du document. Cela m’amène à parler des caractéristiques spécifiques du format \LaTeX{}.

\subsection{\LaTeX{} est un format de rédaction et de structure}

Développons l’analogie avec les logiciels de traitement de texte.
Quand on écrit un texte dans Word ou OpenOffice, on choisit des styles dans un menu et, en les appliquant, on en voit le résultat immédiatement et dans la même fenêtre. Cependant, à l’interne, le document sur lequel on travaille contient ce qu’on appelle des \emph{balises}, comme dans une page Web. Ces balises sont cachées, mais elles sont bel et bien enregistrées dans le document: le moteur de composition de Word ou d’OpenOffice s’en sert pour effectuer la composition du document. Sans ces balises, le logiciel ne saurait pas quelle mise en forme appliquer à quelle partie du texte. Ces balises servent à indiquer non seulement l’apparence du document, mais aussi sa \emph{structure sémantique}, par exemple:

\begin{itemize}
\item Quelles parties du texte sont des titres?
\item Quelles parties sont des citations longues?
\item Quelle partie est une table des matières? Est-elle écrite à la main par l’auteur ou générée automatiquement par le logiciel à partir des titres balisés?
\end{itemize}

Au regard de ces considérations sur le fonctionnement des logiciels de traitement de texte, on peut présenter \LaTeX{} ainsi:

\begin{defex}
\LaTeX{} est au moteur \TeX{} ce que les \emph{styles} d’un logiciel de traitement de texte sont au \emph{moteur de rendu} d’un logiciel de traitement de texte.

L’utilisation du format \LaTeX{} procure au moteur \TeX{} les \emph{instructions} nécessaires à la mise en forme du document et à la désignation de la \emph{structure sémantique} du document.
\end{defex}

\section{Avantages de l’utilisation de \LaTeX{} pour la recherche et la rédaction en philosophie}

Établissons d’abord un aspect primordial du processus de rédaction avec \LaTeX{}:

\begin{defex}
En principe, l’éditeur de texte dans lequel on écrit est indépendant du moteur de composition typographique utilisé.

Je dis bien en principe, car plusieurs éditeurs offrent une certaine \emph{intégration} avec, d’une part, le moteur de composition et, d’autre part, le visualiseur du PDF produit par la composition.
\end{defex}

Cela signifie qu’on a du \emph{choix}. Pour apprendre à utiliser \LaTeX{}, l’éditeur le plus recommandé est \entite{TeXworks} sur Windows ou Linux et \entite{TeXShop} sur Mac~OS~X. Dans tous les cas:

\begin{defex}
On écrit le document dans un fichier de texte simple dans lequel on insère des \emph{commandes} \LaTeX{} et, dans une autre fenêtre qu’on peut placer à côté, on visualise le PDF produit par le moteur \TeX{} à partir du fichier de texte simple. Le fichier de texte simple est donc comme un \emph{plan} servant à indiquer au moteur \TeX{} comment \emph{construire} le document PDF.
\end{defex}

Je montre maintenant les \emph{commandes} dont on a minimalement besoin pour commencer à rédiger un document \LaTeX{}:

\begin{defex}
\begin{Verbatim}[numbers=left,
                 numbersep=4pt]
\documentclass{article}
\usepackage{polyglossia}
\setmainlanguage{french}

\begin{document}

\end{document}
\end{Verbatim}
\end{defex}

Dès qu’on a écrit ça, on peut commencer à écrire le texte entre les commandes \Verb|\begin{document}| et \Verb|\end{document}| et composer le document avec le moteur \entite{XeTeX}, qui est un moteur \TeX{} prenant en charge un très grand nombre de langues anciennes et modernes. Tout ce qui se trouve avant la ligne~5 (\Verb|\begin{document}|) constitue le \emph{préambule} du document. Notez que la ligne~1 de mon exemple indique que la classe \pkgname{article} est choisie: la classe article est aussi celle qui est utilisée pour la version \emph{handout} du présent document. On pourrait aussi indiquer e.g.~\pkgname{ulthese} pour utiliser la classe de l’Université Laval.

Nous allons présenter brièvement différents aspects avantageux de la rédaction dans ce contexte.

\subsection{Prise en charge des langues}

La ligne~2 de mon exemple indique qu’on utilise pour le traitement des langues le paquet~\pkgname{polyglossia}. La ligne suivante indique la langue principale du document: \Verb|\setmainlanguage{french}|. Cela rend le texte du document immédiatement conforme aux règles typographiques traditionnelles: \pkgname{polyglossia} applique même spécifiquement le \opus{Lexique des règles typographiques en usage à l’Imprimerie nationale}, alors on est entre bonnes mains.

Voyons l’exemple d’un texte en grec ancien et de sa traduction, mis côte à côte:

\begin{multicols}{2}
\setlength{\columnseprule}{.1pt}

\begin{otherlanguage}{polutonikogreek}
εὐλογία ἄρα καὶ εὐαρμοστία καὶ εὐσχημοσύνη καὶ εὐρυθμία \stephnum{400ε} εὐηθείᾳ ἀκολουθεῖ, οὐχ ἣν ἄνοιαν οὖσαν ὑποκοριζόμενοι καλοῦμεν ὡς εὐήθειαν, ἀλλὰ τὴν ὡς ἀληθῶς εὖ τε καὶ καλῶς τὸ ἦθος κατεσκευασμένην διάνοιαν.
\end{otherlanguage}

\columnbreak

Ainsi, l’excellence du discours et de l’harmonie, la grâce du geste et du rythme découlent de l’excellence du caractère \stephnum{400e}, non de ce que nous désignons ainsi par euphémisme et qui n’est qu’absence de réflexion, mais au contraire de cette réflexion authentique d’un caractère où s’allient le bien et le beau\autocite[\nopp III,~400d--e]{plat:rep:lrx}.
\end{multicols}

Les deux textes sont composés selon leurs règles typographiques propres. Il a simplement fallu ajouter au préambule du document la ligne suivante:

\begin{defex}
\begin{Verbatim}
\setotherlanguage[variant=poly]{greek}
\end{Verbatim}
\end{defex}

On peut ainsi écrire du grec polytonique en \emph{balisant} les sections en grec avec des commandes appropriées, soit \Verb|\begin{greek}| et \Verb|\end{greek}|.

\subsection{Traitement de la bibliographie}

J’en viens à la partie qui aura peut-être le plus de succès: le traitement automatique de la bibliographie et des références.

On sait qu’il existe un logiciel largement répandu et pour lequel l’Université offre même un certain soutien technique: \entite{EndNote}. Cependant, dans les faits, peu d’étudiants l’utilisent: il est probable qu’une des raisons principales soit le coût du logiciel.

\LaTeX{} comprend deux systèmes bibliographiques, dont l’un est traditionnel et orienté vers les sciences de la nature, l’autre moderne et orienté vers les sciences humaines (l’humble opinion de votre présentateur est que ce dernier est également supérieur pour les sciences de la nature, mais énormément de gens utilisent l’autre système depuis longtemps et passent au nouveau de façon progressive).

Attention: le nom du système traditionnel est Bib\TeX{}, le nom du nouveau système est Bib\LaTeX{}.

\begin{defex}
Bib\LaTeX{} comprend:
\begin{itemize}
\item un \emph{format de rédaction} pour constituer des bases de données bibliographiques;
\item un ensemble de \emph{styles} de citations et de listes des références;
\item un \emph{moteur} qui traite les informations encodées dans la base de données et les met à la disposition de \LaTeX{} pour qu’elles soient typographiées grâce aux \emph{styles} choisis dans le \emph{préambule} du document \LaTeX{}.
\end{itemize}
\end{defex}

Par exemple, dans le préambule du présent document, j’ai écrit:

\begin{defex}
\begin{Verbatim}[numbers=left,
                   numbersep=4pt]
\usepackage[%
    backend=biber,
    style=authortitle,]%
  {biblatex}

\addbibresource{atelier.bib}
\end{Verbatim}
\end{defex}

Ces commandes indiquent que:

\begin{itemize}
\item je désire utiliser le système Bib\LaTeX{},
\item avec le moteur moderne \entite{Biber} (qui prend en charge toutes les langues et tous les jeux de caractères),
\item avec le style \pkgname{authortitle}, qui correspond très bien aux normes en vigueur en sciences humaines à l’Université Laval (plusieurs variantes existent).
\end{itemize}

À la fin de mon document, j’ai écrit:

\begin{defex}
\begin{Verbatim}
\printbibliography
\end{Verbatim}
\end{defex}

Avec cette unique commande, la liste des références est automatiquement imprimée à la fin du document.

Voici à quoi ressemble une entrée dans le fichier de base de données (l’exemple est la \citetitle{plat:rep:lrx} de \citeauthor{plat:rep:lrx} dans l’édition GF qu’on connaît bien):

\begin{defex}
\begin{Verbatim}
@book { plat:rep:lrx,
  author =
    {Platon},
  title =
    {République},
  date =
    {2004},
  editor =
    {Georges Leroux},
  translator =
    {Georges Leroux},
  annotator =
    {Georges Leroux},
  language =
    {french},
  origlanguage =
    {greek},
  edition =
    {deuxième édition corrigée},
  series =
    {GF},
  publisher =
    {Flammarion},
  location =
    {Paris},
  pagetotal =
    {801},
}
\end{Verbatim}
\end{defex}

J’ai une nouvelle réjouissante pour tous: depuis quelques temps, le catalogue Ariane de notre chère bibliothèque rend disponible les données du catalogue au format Bib\TeX{}! Je dis bien Bib\TeX{}, donc le format de l’ancien système (sûrement à cause des préférences des gens en sciences de la nature, qui ont beaucoup influencé l’adoption de \LaTeX{} par la \entite{FESP}), mais les deux formats sont largement compatibles. Il suffit souvent d’ajouter quelques entrées pour rendre les données d’Ariane encore plus complètes.

Enfin, le plus important: la puissance expressive de Bib\LaTeX{} réside dans l’attribution d’une \emph{clé de citation} à chaque ouvrage de la base de données. À la ligne~1 de mon exemple, on peut voir que la clé de citation que j’ai choisie est \Verb|plat:rep:lrx|, ce qui me permet de citer l’ouvrage simplement en écrivant e.g.:

\begin{defex}
\begin{Verbatim}
\autocite[\nopp III,~400d--e]{plat:rep:lrx}
\end{Verbatim}
\end{defex}

à la fin de ma citation. Si j’ai choisi le style \pkgname{authortitle}, cela crée automatiquement une note de bas de page contenant la référence, comme on peut le voir dans l’exemple de texte grec à la section précédente.

L’usage de cette commande a deux autres bénéfices:

\begin{defex}
  \begin{itemize}
  \item La référence complète de l’ouvrage est composée dans la bibliographie à la fin du document, \emph{parce qu’il a été cité au moins une fois dans le document} (on n’a donc pas besoin de réécrire les références en fin de document chaque fois qu’on cite un nouvel ouvrage);
  \item On peut même demander au moteur typographique d’ajouter automatiquement, dans la bibliographie, des \emph{renvois} vers les pages et sections où l’ouvrage est cité. Vous avez bien compris: cela permet de composer automatiquement un \emph{index locorum}!
  \end{itemize}
\end{defex}

\subsection{Balisage sémantique et communication efficace d’une argumentation}

Personnellement, ce que je m’apprête à vous présenter est mon aspect préféré du processus de rédaction avec \LaTeX{}.

Premièrement, sachez que la présentation que vous voyez à l’écran et le \emph{handout} ont été préparés à partir d’un unique document \LaTeX{}. Avec des commandes propres à chaque document, j’ai indiqué quelques paramètres spécifiques pour la présentation à l’écran et pour le \emph{handout}. Aucun copier-coller n’a été nécessaire, car le balisage du texte avec les commandes de \LaTeX{} permet de spécifier la structure et la disposition du texte  \emph{indépendamment l’une de l’autre}, ce qui permet d’obtenir plusieurs documents à partir du même texte.

Cela m’amène à présenter un principe qui est d’une grande importance pour des gens en Philosophie et en Lettres: je veux parler du \emph{balisage sémantique}.

%Les logiciels de traitement de texte ont le mérite de synchroniser la rédaction, la mise en forme et la présentation du document, mais on en paie le prix. Dans les faits, la mise en forme est une tâche ingrate et difficile à maîtriser qui nous détourne de notre occupation principale: la rédaction. En traitement de texte traditionnel, les éléments de la mise en forme ont bien sûr leur raison d’être, mais le problème est le suivant:

% \begin{defex}
%   Les logiciels de traitement de texte nous encouragent à appliquer des éléments de mise en forme afin de nous rappeler du statut particulier d’une partie du texte, mais sans \emph{nommer} la raison spécifique de la distinction visuelle.
% \end{defex}

% Avec un format de rédaction comme \LaTeX{}, le principe est inverse:

\begin{defex}
  \LaTeX{} encourage à \emph{désigner} le sens, le statut ou la fonction d’une partie de texte par \emph{balisage}, c’est-à-dire en encadrant le texte avec des \emph{commandes}.

Le \emph{balisage} peut être défini ainsi:

\begin{quote}
A (document) markup language is a modern system for annotating a document in a way that is syntactically distinguishable from the text\autocite[\nopp version anglaise de \emph{Wikipedia}]{wiki:mrkp-lng}.
\end{quote}

Je traduis ainsi:

Un langage de balisage (documentaire) est un système moderne pour annoter un document de telle sorte que l’annotation puisse être syntaxiquement distinguée du texte.
\end{defex}

Le concept de balisage est au principe du traitement de texte informatique. Du point de vue historique, les premiers langages de balisage étaient surtout destinés à la désignation de l’aspect visuel (typographique) des textes. Cependant, dès les années~60, des chercheurs commencèrent à explorer la possibilité d’utiliser le balisage pour désigner la structure, le sens et la fonction d’un texte et, également, pour en expliciter les \emph{références} et traiter celles-ci de façon automatique. Cela a mené aux concepts d’hypertexte et d’hyper-référence.

Ces recherches ont mené à l’invention parallèle et quasi-simultanée de deux des langages de balisage les plus utilisés aujourd’hui: HTML pour le Web et \LaTeX{} pour l’édition traditionnelle. Tous deux sont, en principe, des langages de balisage \emph{sémantiques}.

\begin{defex}
Un langage de balisage est dit \emph{sémantique} s’il permet de privilégier la désignation de la structure et du sens d’un texte plutôt que la désignation de son aspect visuel. Toutefois, dans la plupart des cas, un aspect visuel donné correspond logiquement à une désignation sémantique donnée.

Par exemple, dans le présent document, la plupart des italiques que vous voyez ont été appliquées \emph{parce que} l’intention de l’auteur est d’accentuer les propos mis en italique. L’aspect visuel correspond donc dans ce cas à une intention rhétorique.

Concrètement, la commande \LaTeX{} qu’on utilise pour signifier cette intention est \Verb|\emph{texte accentué}|. La commande \Verb|\emph{}| \emph{encadre} le texte accentué.
\end{defex}

Il est important de réaliser que \emph{tous} les textes qu’on produit à l’aide d’un ordinateur sont balisés. Les logiciels de traitement de texte ont leur propre format de balisage sous-jacent et tout ce qu’on écrit sur le Web est automatiquement balisé par le système du site Web où l’on écrit (que ce soit un courriel, une publication Twitter ou un formulaire de commande sur un site commercial).

\begin{defex}
Utiliser un système comme \LaTeX{}, c’est donc simplement décider d’écrire soi-même son balisage au lieu de laisser un logiciel l’ajouter automatiquement. Les deux approches ont leurs avantages, mais un langage comme celui de \LaTeX{} permet entre autres les choses suivantes:

\begin{itemize}
\item le contrôle et la maîtrise du texte,
\item la normalisation et la prédictibilité,
\item la communication claire des intentions argumentatives et éditoriales.
\end{itemize}

\end{defex}

J’avance l’idée suivante: pour un philosophe ou un littéraire, il est plus philosophique et plus scientifique de procéder ainsi. Une telle méthodologie de la rédaction est davantage conforme à la nature de nos travaux et de nos enquêtes. Les opérations mentales que demandent un balisage sémantique des textes s’avèrent être bénéfiques pour la compréhension et l’organisation de l’argumentation. Qui plus est, la méthode est on ne peut plus \emph{concrète}: le but immédiat est toujours de produire un texte visible et mis en forme de façon impeccable. Quel meilleur adjuvant dans la communication et la démonstration? J’ose même présenter l’argument suivant:

\begin{defex}
\syllog{La déictique et la démonstration sont des activités qui relèvent de la philosophie.}{Or, un important sous-ensemble des cas d’utilisation, voire \emph{tous} les cas d’utilisation du balisage sémantique sont de nature déictique ou démonstrative.}{Sous ce rapport, le balisage sémantique d’un texte philosophique est en soi une activité philosophique.}

(Ce syllogisme est balisé et mis en forme grâce à une commande fournie par le paquet \LaTeX{} \pkgname{syllogism}.)
\end{defex}

% Concept de déictique (référence, contextualisation, voir Chrysippe): LaTeX permet une *deixis* plus aisée en proposant une syntaxe de *désignation* très compacte et concise. Parler des pratiques contemporaines de référence par hyperlien et accentuation dans les articles de blog ou de sites “Question and Answer”.

% … Il y a la deixis et il y a la lexis, donc l’élément le plus typique et le plus répandu est l’accentuation (emphasis). Or, parfois, l’accentuation ou en général la *désignation de la lexis* coïncide effectivement avec la deixis et avec certains éléments d’analyse linguistique du discours.

% Dès qu’une lexis est bien désignée et différenciée, c’est une victoire et un vecteur d’efficacité pour la communication d’une argumentation. Une telle désignation ne devrait pas être perçue comme purement esthétique (même si son aspect esthétique est lui-même hautement important): il s’agit d’illumination et de démonstration visible de la structure d’un discours argumentatif. Comme le dit Aristote, idéalement la lexis devrait peu importer, mais dans les faits on observe qu’elle est très importante.

% Autre implication: une désignation adéquate facilite la *transformation* du texte. Or, tout texte composé à l’ordinateur est voué à la transformation. Il existe même un champ de l’informatique, soit la transformation de programmes, qui s’applique directement au texte argumentatif tel qu’encodé dans l’ordinateur.

% Exemple: le paquet \pkgname{syllogism}.

\subsection{Publication aisée et normalisation}

En lisant le présent document, vous avez pu constater que la disposition est excellente. Entre autres choses, les césures sont correctement appliquées et la justification est de loin supérieure à celle des logiciels de traitement de texte (et même supérieure au résultat par défaut qu’on peut obtenir avec un logiciel de mise en page comme \entite{Adobe~InDesign}).

Tandis qu’on parle des logiciels de mise en page, je vous donne un \emph{scoop}:

\begin{defex}
Quand on utilise \LaTeX{}, on n’a nullement besoin d’un autre logiciel de mise en page: le document PDF produit par \LaTeX{} est immédiatement conforme aux normes professionnelles d’édition.

On évite ainsi les pénibles labyrinthes du genre:

Auteur~1 en Word~2011 \textrightarrow
Auteur~2 en Word~2010 \textrightarrow
Réviseur en Word~2013 \textrightarrow
Secrétaire d’édition en Adobe InDesign (\textrightarrow)
Auteur~1 encore une fois etc. etc. …
\end{defex}

En comparaison, la formule de la rédaction et de l’édition avec \LaTeX{} est simple et \emph{convainquante}:

\begin{defex}
\begin{equation*}
\frac{(\text{\LaTeX{}} \leftrightarrow \text{\LaTeX{}}) \cdot x}{\mathcal{SCV} + \text{service Web}} = \text{ouvrage \emph{accessible} et \emph{reproductible}}
\end{equation*}

Où:

\begin{itemize}
\item $\mathcal{SCV}$ signifie \emph{système de contrôle des versions};
\item $\leftrightarrow$ représente un échange de différentes versions des documents (différents états de la rédaction ou de la révision);
\item $x$ est le nombre d’auteurs, réviseurs et secrétaires d’édition.
\end{itemize}
\end{defex}

Cet aspect est d’une très grande importance. Si je ne réussis rien d’autre aujourd’hui, j’espère au moins réussir à vous convaincre de \emph{l’urgence} de la normalisation et de la reproductibilité dans le travail documentaire. \LaTeX{} est l’un des meilleurs systèmes pour ce qui est de la normalisation et de la reproductibilité. Un autre bon candidat est LibreOffice/OpenOffice en combinaison avec le format \entite{OpenDocument}, mais je ne pourrai en parler aujourd’hui. Sachez seulement qu’il est primordial d’adopter des formats ouverts, stables, pérennes et maîtrisables quels que soient la discipline et le cas d’utilisation.

En résumé:

\begin{defex}
\begin{description}
\item[matériellement:] le texte est encodé selon les normes d’un format de rédaction à l’aide de mots-clés;
\item[formellement:] la désignation de la structure par un langage formel permet un traitement intelligent, reproductible et maîtrisable et une publication (communication) accessible, efficace et esthétique.
\end{description}
\end{defex}

\section{Que faire pour commencer?}

Si jamais, ô bienheureux public, vous voulez commencer à explorer \LaTeX{}, voici ce que je recommande:

\begin{enumerate}
\item La FESP a publié de l’information sur \LaTeX{} sur son site Web et on peut se procurer le guide d’utilisation de la classe documentaire \pkgname{ulthese} à l’adresse suivante: \\
\url{http://www.ctan.org/pkg/ulthese}
\item Un excellent livre pour apprendre à utiliser \LaTeX{} en science humaines a été récemment publié par un dénommé Maïeul Rouquette (voir la référence complète à la fin du présent document):
  \begin{itemize}
  \item Lien permanent sur Ariane: \\  \url{http://ariane25.bibl.ulaval.ca/ariane/wicket/detail?recherche.lignes[0].type=FK&recherche.lignes[0].champ=a2148913}
  \item Le livre est publié sous licence libre (comme le présent document) et on peut se le procurer sur GitHub: \\ \url{https://github.com/maieul/latexhumain}
  \end{itemize}
\item Vous pouvez vous procurer un système \TeX{} complet — encore une fois, c’est gratuit!
  \begin{itemize}
  \item Sur Windows: \url{http://tug.org/protext/}
  \item Sur Mac~OS~X: \url{http://tug.org/mactex/}
  \item … et alternativement: \TeX{} Live, la distribution \emph{officielle} du \entite{Groupe des Utilisateurs de \TeX{}}, disponible sur tous les systèmes incluant Windows, Mac et Linux: \url{http://tug.org/texlive/} (MacTeX comprend tout ce que \TeX{} Live comprend, plus des utilitaires spécifiques au Mac, mais proTeXt pour Windows n’est pas aussi exhaustif).
  \end{itemize}
\end{enumerate}

\nocite{*}

\printbibliography

\end{document}